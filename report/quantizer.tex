\section{Σύστημα Κβαντιστή \& Αποκβαντιστή}

\par Ένα ακόμα απαραίτητο στοιχείο του συστήματος κωδικοποίησης αποκωδικοποίησης 
είναι το σύστημα κβαντιστής και αποκβαντιστής. Στο σημείο αυτό αρχικά υλοποιούμε 
τη μέθοδο $q = Quant(x, D)$, όπου x είναι ένα δείγμα, D οι στάθμες απόφασης και q 
το σύμβολο στο οποίο αντιστοιχίζεται το δείγμα. Η μέθοδος αυτή είναι πολύ απλή και 
περιγράφεται παρακάτω.


\begin{equation}
q = \left\{
\begin{array}{ll}
0  &    x < D(1) \\
i  & x\in[D(i), D(i+1)) \mbox{ για }  i = 1,2...k-1 \\
k+1 & x \geq D(k)
\end{array}
\right
\end{equation}

\par Στη συνέχεια θα δημιουργήσουμε ένα σύστημα αποκβάντισης, δηλαδή μία μέθοδος 
στην οποία δίνοντας ένα σύμβολο και τις στάθμες κβαντισμού μας επιστρέφει την 
αποκβαντισμένη τιμή του συμβόλου. Η συνάρτηση αυτή είναι η $x = iQuant(q, L)$, όπου 
q το σύμβολο, L οι στάθμες κβαντισμού και x η τιμή που επιστρέφει. Ουσιαστικά αυτό 
που κάνει η μέθοδος αυτή είναι να δίνει στο x την τιμή του L που αντιστοιχεί για 
σύμβολο q, δηλαδή $x = L(q)$.

\par Το τελευταίο που μένει για να έχουμε ένα πλήρες σύστημα είναι η δημιουργία 
του κβαντιστή, δηλαδή μία μεθόδου που θα μας δημιουργεί τo L και το D. Στην 
περιπτωσή μας θα δημιουργήσουμε έναν ομοιόμορφο κβαντιστή με τη μέθοδο $[D, L] = 
quantLevels(n, xmin, xmax)$, όπου n ο επιθυμητός αριθμός bit ανά σύμβολο. Άρα 
θα έχουμε $k = 2^n$ στάθμες κβαντισμού. Άρα το διάνυσμα D θα έχει k-1 στοιχεία 
και το L k. Στη συνέχεια υπολογίζουμε το $\Delta = \frac{x_{max}-x_{min}}{k}$. 
Τώρα εύκολα βρίσκουμε το D ως $D(i)=x_{min}+i \times \Delta, i=1...k-1$ και το 
L ως $L(i) = x_{min} + \Delta/2 + (i-1)\times\Delta, i = 1...k$.

\par Για να ελένξουμε κατα πόσο είναι σωστές οι υλοποιήσεις που έχουμε κάνει χρησιμοποιούμε 
την συνάρτηση testQ3 από την οποία παίρνουμε pass.

