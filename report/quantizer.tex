\section{Σύστημα Κβαντιστή \& Αποκβαντιστή}

\par Ένα ακόμα απαραίτητο στοιχείο του συστήματος κωδικοποίησης αποκωδικοποίησης
είναι η υλοποίηση ενός κβαντιστή και ενός αποκβαντιστή. Στο σημείο αυτό αρχικά υλοποιούμε
τη μέθοδο:
\begin{lstlisting}[style=myMatlab]
  function q = Quant(x, D)
\end{lstlisting}
\noindent του κβαντιστή, όπου x είναι ένα δείγμα, D οι στάθμες απόφασης και q
το σύμβολο στο οποίο αντιστοιχίζεται το δείγμα. Η λειτουργία αυτής της συνάρτησης είναι πολύ απλή και
περιγράφεται από την παρακάτω εξίσωση:

\begin{equation}
q =
\begin{cases}
0  &    x < D(1) \\
i  & x\in[D(i), D(i+1)) \mbox{ για }  i = 1,2...k-1 \\
k+1 & x \geq D(k)
\end{cases}
\end{equation}

\par Στη συνέχεια θα δημιουργήσουμε έναν αποκβαντιστή, δηλαδή μία μέθοδος
στην οποία δίνοντας ένα σύμβολο και τις στάθμες κβαντισμού μας επιστρέφει την
αποκβαντισμένη τιμή του συμβόλου. Η συνάρτηση αυτή είναι η
\begin{lstlisting}[style=myMatlab]
  function x = iQuant(q, L)
\end{lstlisting}
\noindent όπου
q το σύμβολο, L οι στάθμες κβαντισμού και x η τιμή που επιστρέφει. Ουσιαστικά αυτό
που κάνει η μέθοδος αυτή είναι να δίνει στο x την τιμή του L που αντιστοιχεί για
σύμβολο q, δηλαδή $x = L(q)$.

\par Το τελευταίο που μένει για να έχουμε ένα πλήρες σύστημα είναι η δημιουργία
των παραμέτρων του κβαντιστή και του κβαντιστή, δηλαδή μίας μεθόδου που θα μας δημιουργεί τo L,
δηλαδή τις στάθμες κβαντισμού, και το D, τις περιοχές απόφασης. Στην
περίπτωση μας, θα δημιουργήσουμε έναν ομοιόμορφο κβαντιστή με την μέθοδο:
\begin{lstlisting}[style=myMatlab]
  function [D, L] = quantLevels(n, xmin, xmax)
\end{lstlisting}
, όπου n ο επιθυμητός αριθμός bit ανά σύμβολο. Άρα
θα έχουμε $k = 2^n$ στάθμες κβαντισμού, το διάνυσμα D θα έχει k-1 στοιχεία
και το L \emph{k} στοιχεία. Στη συνέχεια υπολογίζουμε το διάστημα της διαμέρισης:
\begin{equation}
  \label{eq:quant_delta}
  \Delta = \frac{x_{max}-x_{min}}{k}
\end{equation}
Με βάση το παραπάνω, εύκολα βρίσκουμε τις περιοχές απόφασης \emph{D} ως:
\begin{equation}
  \label{eq:quant_regions}
  \bm{\boxed{D(i)=x_{min}+i \cdot \Delta}} \;, \quad i=1, \ldots,k-1
\end{equation}
\noindent και τις στάθμες κβαντισμού \emph{L}:
\begin{equation}
  \label{eq:quant_levels}
  \bm{\boxed{L(i) = x_{min} + \Delta/2 + (i-1)\cdot \Delta}} \; , \quad i=1, \ldots,k-1
\end{equation}

\par Για να ελέγξουμε κατά πόσο είναι σωστές οι υλοποιήσεις που έχουμε κάνει χρησιμοποιούμε
την συνάρτηση testQ3 από την οποία παίρνουμε pass.

